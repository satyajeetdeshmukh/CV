%%%%%%%%%%%%%%%%%%%%%%%%%%%%%%%%%%%%%%%%%
% Medium Length Professional CV
% LaTeX Template
% Version 2.0 (8/5/13)
%
% This template has been downloaded from:
% http://www.LaTeXTemplates.com
%
% Original author:
% Trey Hunner (http://www.treyhunner.com/)
%
% Important note:
% This template requires the resume.cls file to be in the same directory as the
% .tex file. The resume.cls file provides the resume style used for structuring the
% document.
%
%%%%%%%%%%%%%%%%%%%%%%%%%%%%%%%%%%%%%%%%%

%----------------------------------------------------------------------------------------
%	PACKAGES AND OTHER DOCUMENT CONFIGURATIONS
%----------------------------------------------------------------------------------------

\documentclass{resume} % Use the custom resume.cls style
%\documentclass{article}
\usepackage{fontawesome}
\usepackage[left=0.75in,top=0.6in,right=0.75in,bottom=0.6in]{geometry} % Document margins
\newcommand{\tab}[1]{\hspace{.2667\textwidth}\rlap{#1}}
\newcommand{\itab}[1]{\hspace{0em}\rlap{#1}}
\name{SATYAJEET DESHMUKH}
\address{ \faGithub \hspace{2px} GitHub.com/SatyajeetDeshmukh }
\address{ \faPhone \hspace{2px} (+91)7263942220 \hspace{3px} \faEnvelope \hspace{2px} ee170002042@iiti.ac.in}
\begin{document}


%----------------------------------------------------------------------------------------
%	EDUCATION SECTION
%----------------------------------------------------------------------------------------

\begin{rSection}{Education}

{\bf Indian Institute of Technology Indore} \hfill {\em July 2017 - Present} 
\\ Junior Undergraduate (3rd Year) \hfill { Overall CPI: 8.48/10}
\\ Department of Electrical Engineering \hfill { (upto 5th Semester) }


\end{rSection}
%----------------------------------------------------------------------------------------
%	TECHNICAL STRENGTHS SECTION
%----------------------------------------------------------------------------------------

\begin{rSection}{Skills}

\begin{tabular}{ @{} >{\bfseries}l @{\hspace{2em}\vspace{0.3em}} l }
Electrical:  &  Analog, Digital, Power Electronics, PCB Design, Lab Hardware Testing  \\
Programming: &  DS and Algorithms, C, C++, Python, Matlab, Git \\
Other: & Web Design, Bash, LaTeX, German, Basic Finance \\
\end{tabular}

\end{rSection}

%----------------------------------------------------------------------------------------
%	WORK EXPERIENCE SECTION
%----------------------------------------------------------------------------------------

\begin{rSection}{Experience} 
All Project Files are available on GitHub.

\begin{rSubsection}{Power Electronics PCB Design}{March 2019 - August 2019}{Summer Project under Dr. Amod Umarikar}{IIT Indore}
\item Deployed a closed-loop three mode (Buck, Boost, Buck-Boost) SMPS of rating 100 watt which used a PI controller.
\item Involved use of simulation tools and then final hardware testing after build completion.
\end{rSubsection}

\begin{rSubsection}{Object Tracking}{May-June 2018}{Part of Robocon 2018}{IIT Indore}
\item Program to track the position of the ball and give a trail as well as estimate the distance of the ball from webcam using given focal length.
\end{rSubsection}



\begin{rSubsection}{Transmission Lines Matlab GUI}{April 2019}{Course Project under Dr. Saptarshi Ghosh}{IIT Indore}
\item Made a GUI for plotting voltage as a function of time and space in Matlab based on theoretical model of transmission lines.
\end{rSubsection}

\begin{rSubsection}{Simulation of BMS}{August 2018}{Round 1 of Shell Eco-marathon}{IIT Indore}
\item A simple simulation of a Battery Management System using Matlab and Simulink models.
\end{rSubsection}

\begin{rSubsection}{Signals and Systems Frequency Analysis in Matlab}{April 2019}{Course Project under Dr. Ram Bilas Pachori}{IIT Indore}
\item A simple Matlab signal processing project involving use of FFT and filter design.
\end{rSubsection}

\end{rSection}
%----------------------------------------------------------------------------------------
\begin{rSection}{Technical Interests}
\begin{rSubsection}{}{}{}{}
\item Power Electronics
\item Control Theory
\item Embedded Systems
\end{rSubsection}
\end{rSection}

%	EXAMPLE SECTION
%----------------------------------------------------------------------------------------

\begin{rSection}{Academic Record}
\begin{rSubsection}{}{}{}{}
\item Ranked 3770 among 160 thousand students in JEE Advanced 2017 who were eligible from 1.2 million students who took the JEE Main.
\item Scored 83.38\% in 12th HSC in 2017 and 91.80\% in 10th SSC in 2015, both Maharashtra state boards.
\end{rSubsection}
\end{rSection}

%----------------------------------------------------------------------------------------
\begin{rSection}{Relevant Courses Taken}
\itab{\textbf{Core Courses}} 						\tab{}  \tab{\textbf{Math Courses}}
\\ \itab{Electronic Devices}						 \tab{} \tab{Real Analysis} 
\\ \itab{Power Electronics}							 \tab{}  \tab{Linear Algebra and Differential Equations I} 
\\ \itab{Signals and Systems}						\tab{}  \tab{Complex Analysis and Differential Equations II}
\\ \itab{Analog Circuits}			 				\tab{} \tab{Numerical Methods}
\\ \itab{VLSI Systems \& Technology}					\tab{}  \tab{Probability and Random Processes} 
\\ \itab{Microprocessors \& Digital Systems Design}			\tab{} \tab{}
\\ \itab{Control Systems*}							\tab{} \tab{}
\\ \itab{Digital Signal Processing*}					\tab{} \tab{}
\\ \itab{Digital Communications*}						\tab{} \tab{}
\\ \itab{Power Systems*}							\tab{} \tab{}
\item * ongoing
\end{rSection}

\end{document}
